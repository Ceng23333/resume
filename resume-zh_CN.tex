% !TEX TS-program = xelatex
% !TEX encoding = UTF-8 Unicode
% !Mode:: "TeX:UTF-8"

\documentclass{resume}
\usepackage{zh_CN-Adobefonts_external}
\usepackage{linespacing_fix}
\usepackage{cite}
\usepackage{xcolor}
\usepackage{enumitem}

% 调整间距以压缩版面
\setlist[itemize]{leftmargin=*,itemsep=0.3em,parsep=0.1em}  % 减小item间距
\setlist[itemize,1]{labelindent=\parindent}  
\setlist[itemize,2]{leftmargin=1em,itemsep=0.2em,parsep=0.1em}  % 减小subitem间距

% 定义主题颜色
\definecolor{primary}{RGB}{70, 130, 180}
\definecolor{secondary}{RGB}{120, 120, 120}

% 减小分隔线上下间距
\newcommand{\mysep}{
  \vspace{2pt}  % 从4pt减少到2pt
  {\color{primary}\hrule height 0.8pt}
  \vspace{4pt}  % 从6pt减少到4pt
}

\begin{document}
\pagenumbering{gobble}

% 减小姓名和基本信息之间的间距
{\Huge\textbf{曾华}}\vspace{-2pt}

\basicInfo{
  \textcolor{secondary}{
    \email{thsszh@hotmail.com} \textperiodcentered\ 
    \phone{(+86) 156-5275-9890} \textperiodcentered\ 
    \linkedin[HuaZeng]{https://www.linkedin.com/in/hua-zeng-b28147299/}
    \github[Ceng23333]{https://github.com/Ceng23333}
  }
}

% 减小section之间的间距
\section{\faGraduationCap\ 教育背景}
\datedsubsection{\textbf{清华大学}, 北京}{2016 -- 2019}
\textit{工学硕士}\ 软件工程\vspace{-2pt}
\datedsubsection{\textbf{清华大学}, 北京}{2012 -- 2016}
\textit{工学学士}\ 计算机软件

\section{\faBriefcase\ 工作及项目经历}
\datedsubsection{\textbf{北京数元灵科技有限公司}, 北京}{2022 -- 至今}
\role{架构开发工程师}{大数据部门}
\begin{onehalfspacing}
\textbf{数据湖仓开源项目开发和运维} \textit{LakeSoul}: \url{https://github.com/lakesoul-io/LakeSoul}
\begin{itemize}
  \item 数据湖仓Native内核开发
  \subitem{项目背景: 传统数据湖仓架构与特定计算引擎强耦合,导致跨平台开发成本高昂,亟需一个统一的核心架构}
  \subitem{项目成果: 设计并实现基于Apache Arrow的跨语言Native内核,统一数据湖仓的元数据管理和数据访问接口。支持Java、Python、Rust等多语言生态,实现与Spark、Flink、Doris等主流计算引擎的无缝集成并介入了S3和HDFS等主流存储系统,显著提升了系统的扩展性和互操作性}
  \item LakeSoul数据源接入Apache Doris
  \subitem{项目背景: 为满足数据湖仓场景下的高性能分析需求,需要将LakeSoul与高性能OLAP引擎Apache Doris进行深度整合}
  \subitem{项目成果: 完成LakeSoul数据源与Apache Doris的深度集成,实现数据的高效读取和分析。设计并发布基于Docker的一站式部署方案,显著降低了用户的使用门槛,获得社区积极反馈}
  \item 基于Local Sensitive Hashing的数据湖仓向量化解决方案
  \subitem{项目背景: 随着Data+AI场景的兴起,传统向量化技术难以满足数据湖仓的使用场景需求,亟需创新性的解决方案}
  \subitem{项目成果: 将LSH技术应用于数据湖仓向量化场景,设计并实现了高效的向量索引和检索方案。在多个标准数据集上的测试验证该方案的可行性}
\end{itemize}
\end{onehalfspacing}

\begin{onehalfspacing}
\textbf{业务项目支持}
\begin{itemize}
  \item Apache Flink CDC整库同步解决方案
  \subitem{项目背景: 企业级数据同步场景对效率和兼容性要求极高,整库同步相比单表同步能显著提升资源利用效率,但对异构数据源的支持提出了更高要求}
  \subitem{项目成果: 设计并实现基于Apache Flink CDC的企业级整库同步方案,支持MySQL、PostgreSQL、Oracle等主流数据库,实现准实时延迟的数据同步,同时保证数据一致性}
  \item Apache Arrow格式接入数据湖仓的Flink/Java生态支持
  \subitem{项目背景: 特定业务场景需要更高效的数据接入方式,原有的基于计算引擎的方案无法满足性能要求}
  \subitem{项目成果: 开发基于Apache Arrow的高性能数据接入方案,包括Apache Flink Connector和原生Java SDK。显著提升数据接入效率,降低了系统资源开销}
  \item 基于LSM Tree的数据湖仓数据压缩解决方案
  \subitem{项目背景: 面对业务场景的巨大数据增量,传统全量压缩方案在性能和资源消耗上难以满足需求}
  \subitem{项目成果: 将LSM Tree思想应用于数据压缩,实现增量压缩和分层压缩。该方案显著提升了压缩效率,降低存储成本,同时保证了查询性能}
\end{itemize}
\end{onehalfspacing}

\datedsubsection{\textbf{北京阿里巴巴科技有限公司}, 北京}{2019 -- 2022}
\role{高级开发工程师}{阿里妈妈算法平台团队}
\textbf{广告部门算法中台开发}
\begin{onehalfspacing}
\begin{itemize}
  \item 广告推荐业务样本实时数据流框架开发和运维
  \subitem{项目背景: 为支持实时个性化广告推荐,需要构建低延迟、高可靠的样本数据处理框架}
  \subitem{项目成果: 基于Apache Flink设计实现了端到端的实时样本处理框架,支持实时延迟的数据处理,实现在线学习模型的实时更新,显著提升推荐效果}
  \item 广告推荐业务深度学习模型统一训练范式
  \subitem{项目背景: 集团内存在不同的训练框架,业务方迁移业务的成本高,需要统一训练流程以提升效率}
  \subitem{项目成果: 结合广告推荐业务场景,设计并实现基于TensorFlow的统一训练框架,提供标准化的模型训练接口和工具链。成功整合多个业务团队的训练流程,显著提升了开发效率}
  \item 广告搜索推荐样本云原生数据湖仓建设
  \subitem{项目背景: 需要构建统一的数据存储方案,确保实时与离线样本系统的数据一致性,同时降低运维成本}
  \subitem{项目成果: 设计并落地云原生数据湖仓解决方案,实现离线和实时数据的统一管理。该方案显著提升了数据质量,降低了运维成本}
\end{itemize}
\end{onehalfspacing}

% IT技能部分压缩为两行
\section{\faCogs\ IT 技能}
\begin{itemize}[itemsep=0.2em]
  \item \textbf{编程语言:} Rust, Java, Python \hspace{2em} \textbf{操作系统:} MacOS, Linux
  \item \textbf{开源框架:} Apache Flink, Apache Spark, Apache Arrow, Apache DataFusion, Substrait, Apache Doris, LakeSoul, Kubenetes, Faiss, TensorFlow, Apache Hive, Apache Iceberg, Apache Hudi
\end{itemize}

\section{\faHeartO\ 获奖情况与学术成果}
\datedline{全国中学生奥林匹克信息学竞赛, \textit{银牌,入选国家集训队}}{2011 年}
\datedline{清华大学挑战杯, \textit{二等奖}}{2014 年}
\datedline{清华大学科学创新优秀奖学金}{2014 年}
\datedline{《使用特征空间归一化主类距离的智能零售场景开放集分类算法》计算机图形学报}{2014 年}
\datedline{《数据湖上数据更新与读取的方法及相关设备》 发明专利号: 202311267583X}{2023 年}

\section{\faInfo\ 其他}
\begin{itemize}
  \item MBTI: INTJ
  \item 人生目标: 健康长寿, 终身学习
  \item 生产力工具: Obsidian, Notion, Cursor, GitHub, NotebookLM, DouBao
  \item 语言能力: 粤语——母语, 普通话——母语, 英语——日常交流, 日语——入门
\end{itemize}

\end{document}
